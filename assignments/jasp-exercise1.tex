\documentclass[11pt,]{article}
\usepackage[margin=1in]{geometry}
\newcommand*{\authorfont}{\fontfamily{phv}\selectfont}
\usepackage[]{mathpazo}
\usepackage{abstract}
\renewcommand{\abstractname}{}    % clear the title
\renewcommand{\absnamepos}{empty} % originally center
\newcommand{\blankline}{\quad\pagebreak[2]}

\providecommand{\tightlist}{%
  \setlength{\itemsep}{0pt}\setlength{\parskip}{0pt}} 
\usepackage{longtable,booktabs}

\usepackage{parskip}
\usepackage{titlesec}
\titlespacing\section{0pt}{12pt plus 4pt minus 2pt}{6pt plus 2pt minus 2pt}
\titlespacing\subsection{0pt}{12pt plus 4pt minus 2pt}{6pt plus 2pt minus 2pt}

\usepackage{titling}
\setlength{\droptitle}{-.25cm}

%\setlength{\parindent}{0pt}
%\setlength{\parskip}{6pt plus 2pt minus 1pt}
%\setlength{\emergencystretch}{3em}  % prevent overfull lines 

\usepackage[T1]{fontenc}
\usepackage[utf8]{inputenc}
\linespread{1.05}

\usepackage{fancyhdr}
\pagestyle{fancy}
\usepackage{lastpage}
\renewcommand{\headrulewidth}{0.3pt}
\renewcommand{\footrulewidth}{0.0pt} 
\lhead{\footnotesize \textbf{Dr.~Furtado}}
\chead{}
\rhead{\footnotesize \emph{KIN 610 \textbar{} JASP Assignment 1}}
%\lfoot{}
%\cfoot{\small \thepage/\pageref*{LastPage}}
%\rfoot{}

\fancypagestyle{firststyle}
{
\renewcommand{\headrulewidth}{0pt}%
   \fancyhf{}
   \fancyfoot[C]{\small \thepage/\pageref*{LastPage}}
}

%\def\labelitemi{--}
%\usepackage{enumitem}
%\setitemize[0]{leftmargin=25pt}
%\setenumerate[0]{leftmargin=25pt}


\usepackage{titlesec}

\titleformat*{\subsection}{\itshape}

\makeatletter
\@ifpackageloaded{hyperref}{}{%
\ifxetex
  \usepackage[setpagesize=false, % page size defined by xetex
              unicode=false, % unicode breaks when used with xetex
              xetex]{hyperref}
\else
  \usepackage[unicode=true]{hyperref}
\fi
}
\@ifpackageloaded{color}{
    \PassOptionsToPackage{usenames,dvipsnames}{color}
}{%
    \usepackage[usenames,dvipsnames]{color}
}
\makeatother
\hypersetup{breaklinks=true,
            bookmarks=true,
            pdfauthor={ ()},
             pdfkeywords = {},  
            pdftitle={KIN 610 \textbar{} JASP Assignment 1},
            colorlinks=true,
            citecolor=blue,
            urlcolor=blue,
            linkcolor=magenta,
            pdfborder={0 0 0}}
\urlstyle{same}  % don't use monospace font for urls


\setcounter{secnumdepth}{0}





\usepackage{setspace}

\title{KIN 610 \textbar{} JASP Assignment 1}
\author{Dr.~Furtado}
\date{}


\def\citeapos#1{\citeauthor{#1}'s (\citeyear{#1})}

% header includes!
\linespread{1.05}


\begin{document}  



\thispagestyle{plain} 

\begin{flushleft}\Large \bf KIN 610 \textbar{} JASP Assignment
1  \end{flushleft}
	\vspace{1 mm}   
Dr.~Furtado \\
\emph{Department of Kinesiology, Cal State Northridge} \\
\texttt{\href{mailto:ovandef@csun.edu}{\nolinkurl{ovandef@csun.edu}}}   \\

% \blankline 
  

\hrule

\vspace{6 mm}
	


\hypertarget{learning-objectives}{%
\section{Learning objectives}\label{learning-objectives}}

\begin{enumerate}
\def\labelenumi{\arabic{enumi}.}
\tightlist
\item
  Differentiate between independent and dependent variables
\item
  Assign the correct data level to variables (nominal, ordinal, and
  scale)
\item
  Create and compute new variables in JASP
\item
  Select and unselect cases in JASP
\item
  Create and interpret histograms, boxplots, and Q-Q Plots
\end{enumerate}

\hypertarget{dataset-link}{%
\section{Dataset: Link}\label{dataset-link}}

\begin{enumerate}
\def\labelenumi{\arabic{enumi}.}
\tightlist
\item
  Open the data set in JASP (Open \textgreater{} Computer)
\item
  Note the data set has 3 variables
\item
  Set the variables to their appropriate scale (e.g., Nominal, Ordinal,
  Scale)
\end{enumerate}

\hypertarget{questions}{%
\section[Questions ]{\texorpdfstring{Questions
\footnote{Each question is worth 2 points}}{Questions }}\label{questions}}

\hypertarget{question-1-variables}{%
\subsection{\texorpdfstring{\textbf{Question 1}:
Variables}{Question 1: Variables}}\label{question-1-variables}}

What is(are) the independent variable(s) for this data set?

What is(are) the dependent variable(s) for this data set?

\begin{center}\rule{0.5\linewidth}{0.5pt}\end{center}

\hypertarget{question-2-create-a-new-scale-variable-and-name-it-log10-heart-rate-then-compute-the-log10-of-heart-rate-into-this-variable.-then-calculate-the-min-max-and-mean-for-this-variable.}{%
\subsection{\texorpdfstring{\textbf{Question 2}: Create a new
\textbf{scale} variable and name it \textbf{log10-heart-rate}; then
compute the log10 of \textbf{Heart Rate} into this variable. Then,
calculate the min, max, and mean for this
variable.}{Question 2: Create a new scale variable and name it log10-heart-rate; then compute the log10 of Heart Rate into this variable. Then, calculate the min, max, and mean for this variable.}}\label{question-2-create-a-new-scale-variable-and-name-it-log10-heart-rate-then-compute-the-log10-of-heart-rate-into-this-variable.-then-calculate-the-min-max-and-mean-for-this-variable.}}

\begin{center}\rule{0.5\linewidth}{0.5pt}\end{center}

\hypertarget{question-3-select-cases-so-that-only-female-runners-are-selected-for-the-current-analysis.-then-using-the-variable-heart-rate-create-a-histogram-for-the-selected-cases.-without-considering-the-other-sources-of-normality-e.g.-zkurt-zskew-q-q-plots-shapiro-wilk-test-does-the-distribution-of-scores-appear-to-be-approximating-or-deviating-from-normality}{%
\subsection{\texorpdfstring{\textbf{Question 3}: Select cases so that
\textbf{only} Female runners are selected for the current analysis.
Then, using the variable \textbf{Heart Rate}, create a histogram for the
selected cases. Without considering the other sources of normality
(e.g., zkurt, zskew, q-q plots, Shapiro-Wilk test), does the
distribution of scores appear to be approximating or deviating from
normality?}{Question 3: Select cases so that only Female runners are selected for the current analysis. Then, using the variable Heart Rate, create a histogram for the selected cases. Without considering the other sources of normality (e.g., zkurt, zskew, q-q plots, Shapiro-Wilk test), does the distribution of scores appear to be approximating or deviating from normality?}}\label{question-3-select-cases-so-that-only-female-runners-are-selected-for-the-current-analysis.-then-using-the-variable-heart-rate-create-a-histogram-for-the-selected-cases.-without-considering-the-other-sources-of-normality-e.g.-zkurt-zskew-q-q-plots-shapiro-wilk-test-does-the-distribution-of-scores-appear-to-be-approximating-or-deviating-from-normality}}

Note: If turning this assignment for grade, \textbf{copy and paste the
histogram below} and \textbf{provide your answers} below the graph.

\begin{center}\rule{0.5\linewidth}{0.5pt}\end{center}

\hypertarget{question-4-before-proceeding-unselect-cases-so-that-both-groups-e.g.-runners-and-control-can-be-considered-for-the-this-analysis.-using-the-variable-you-created-log10-heart-rate-create-boxplots-for-both-groups-so-that-they-appear-side-by-side-on-the-same-graph.-now-inspect-both-boxplots-and-list-potential-outliers-for-each-group-control-and-runners.}{%
\subsection{\texorpdfstring{\textbf{Question 4}: Before proceeding,
unselect cases so that both groups (e.g., runners and control) can be
considered for the this analysis. Using the variable you created
(log10-heart-rate), create boxplots for both groups so that they appear
side-by-side on the same graph. Now, inspect both boxplots and list
potential outliers for each group (control and
runners).}{Question 4: Before proceeding, unselect cases so that both groups (e.g., runners and control) can be considered for the this analysis. Using the variable you created (log10-heart-rate), create boxplots for both groups so that they appear side-by-side on the same graph. Now, inspect both boxplots and list potential outliers for each group (control and runners).}}\label{question-4-before-proceeding-unselect-cases-so-that-both-groups-e.g.-runners-and-control-can-be-considered-for-the-this-analysis.-using-the-variable-you-created-log10-heart-rate-create-boxplots-for-both-groups-so-that-they-appear-side-by-side-on-the-same-graph.-now-inspect-both-boxplots-and-list-potential-outliers-for-each-group-control-and-runners.}}

Note: If turning this assignment for grade, \textbf{copy and paste the
boxplot graph below} and \textbf{provide your answers} below the graph.

\begin{center}\rule{0.5\linewidth}{0.5pt}\end{center}

\hypertarget{question-5-before-proceeding-unselect-all-previously-selected-cases.-using-the-variable-heart-rate-create-two-q-q-plots-one-for-males-and-one-for-females.-without-considering-the-other-sources-of-normality-e.g.-zkurt-zskew-histogram-shapiro-wilk-test-does-the-distribution-of-scores-appear-to-be-approximating-or-deviating-from-normality-for-males-how-about-females}{%
\subsection{\texorpdfstring{\textbf{Question 5}: Before proceeding,
unselect \textbf{all} previously selected cases. Using the variable
\emph{Heart Rate}, create two Q-Q Plots (one for males and one for
females). Without considering the other sources of normality (e.g.,
zkurt, zskew, histogram, Shapiro-Wilk test), does the distribution of
scores appear to be approximating or deviating from normality for males?
How about
females?}{Question 5: Before proceeding, unselect all previously selected cases. Using the variable Heart Rate, create two Q-Q Plots (one for males and one for females). Without considering the other sources of normality (e.g., zkurt, zskew, histogram, Shapiro-Wilk test), does the distribution of scores appear to be approximating or deviating from normality for males? How about females?}}\label{question-5-before-proceeding-unselect-all-previously-selected-cases.-using-the-variable-heart-rate-create-two-q-q-plots-one-for-males-and-one-for-females.-without-considering-the-other-sources-of-normality-e.g.-zkurt-zskew-histogram-shapiro-wilk-test-does-the-distribution-of-scores-appear-to-be-approximating-or-deviating-from-normality-for-males-how-about-females}}

Note: If turning this assignment for grade, \textbf{copy and paste the
Q-Q Plots below} and \textbf{provide your answers} below the graphs.




\end{document}

\makeatletter
\def\@maketitle{%
  \newpage
%  \null
%  \vskip 2em%
%  \begin{center}%
  \let \footnote \thanks
    {\fontsize{18}{20}\selectfont\raggedright  \setlength{\parindent}{0pt} \@title \par}%
}
%\fi
\makeatother
